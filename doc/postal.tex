\documentclass[12pt]{article}
\usepackage[
    pdftitle={Postal},
    pdfauthor={Christian Hergert},
    pdfsubject={Postal},
    pdfkeywords={Programming},
    bookmarks, bookmarksopen,
    pdfstartview=FitH,
    colorlinks,linkcolor=blue,citecolor=blue,
    urlcolor=red,
]{hyperref}
\usepackage{color}
\usepackage{listings}
\usepackage{parskip}
\usepackage{graphicx}
\usepackage{tikz}
\usepackage[light,math]{iwona}
\usepackage[T1]{fontenc}
\parskip 7.2pt

\newcommand{\versal}[1]{\noindent{\Huge #1\kern-.10em}}
\newcommand{\file}[1]{{\bf\ttfamily #1}}
\newcommand{\ident}[1]{{\it\ttfamily #1}}
\newcommand{\shell}[1]{{\it\ttfamily #1}}
\newcommand{\python}[1]{{\it\ttfamily #1}}
\newcommand{\ruby}[1]{{\it\ttfamily #1}}
\newcommand{\book}[2]{{\it\ttfamily #1} by {\it #2}}
\newcommand{\program}[1]{\it\ttfamily #1}

\lstnewenvironment{code}[1]
{
    \lstset{language=C,
            showstringspaces=false,
            basicstyle=\ttfamily\small,
            caption=#1,
            numbers=left}
}
{}


\lstnewenvironment{Terminal}
{
    \lstset{basicstyle=\ttfamily\small,
            breaklines=true}
}
{}


\title{Postal \\ Push Notification Server}
\author{Christian Hergert \\
    \small \texttt{christian@catch.com}
}

\date{November 2012}

\begin{document}
\maketitle

Postal is a free and open source push notification server written in C.

\section{Introduction}

Postal is a Push Notification Server.
It supports device management and delivery of push notifications to \emph{Apple Push Notification Service (APS)}, \emph{Google Cloud to Device Messaging (C2DM)}, and \emph{Google Cloud Messaging (GCM)}.
These are the services currently supported by the Apple iOS\footnote{Apple and iOS are registered trademarks of Apple, Inc.} and Google Android\footnote{Google and Android are registered trademarks of Google.} mobile device platforms.

Postal is a server that you run as part of your infrastructure.
You interact with it over an HTTP REST API.
It stores and manages devices using MongoDB\footnote{For more information on MongoDB, visit http://www.mongodb.org.}.
Optionally, you can receive events about events within Postal via the publish-subscribe mechanism of Redis\footnote{For more information on Redis, visit http://redis.io}.

\section{Installation}

Postal is designed to run on GNU/Linux, however other operating systems may be supported.

\subsection{Installing from Source}

You can fetch the most recent version of Postal from \url{https://github.com/catch/postal/downloads}.
Installation follows the typical procedures for Free and Open Source software on GNU/Linux.

Many of the configuration options seen below are not required but illustrate their use. See \verb|./configure --help| for more options.

\begin{figure}[h!]
\hrule
\begin{Terminal}
tar --lzma -xf postal-0.2.0.tar.xz
cd postal-0.2.0
./configure --prefix=/usr --libdir=/usr/lib64 --sysconfdir=/etc/postal --enable-redis=yes --enable-debug=minimal --enable-trace=no
make
sudo make install
\end{Terminal}
\hrule
\caption{Installing Postal}
\end{figure}

\subsection{Debian}

Debian packages are not available at this time, but are planned.

\subsection{Ubuntu 12.04 LTS}

Ubuntu packages are not available at this time, but are planned.

\subsection{Fedora 18}

Fedora packages are not available at this time, but are planned.

\section{Configuration}

Postal uses a standard \verb|key-file| format for configuration.
This is similar to an \verb|.ini| format from other Operating Systems.

Each configuration section is provided below with their options.

\subsection{http}

This section provides configuration of the HTTP REST API.
The embedded HTTP is how external systems communicate with Postal.

\begin{figure}[h!]
\centering
\begin{tabular}{l l l}
\hline
Option & Value & Description \\
\hline
nologging & \verb|true| or \verb|false| & If HTTP logging is disabled. \\
port & \verb|5300| & The TCP port to listen on. \\
logfile & \file{/var/log/postal/access.log} & The HTTP access logfile. \\
\hline
\end{tabular}
\caption{HTTP Configuration Options}
\end{figure}

See the example below.

\begin{figure}[h!]
\begin{Terminal}
[http]
nologging = false
port = 8080
logfile = /var/log/postal/access.log
\end{Terminal}
\caption{HTTP Configuration Example}
\end{figure}

\subsection{mongo}

This section provides configuration for communicating with MongoDB.
You are responsible for configuring and managing your MongoDB server.

It would be wise to have a \emph{replicaSet} configured to prevent loosing data!

\begin{figure}[h!]
\centering
\begin{tabular}{l l l}
\hline
Option & Value & Description \\
\hline
db & \emph{myapp} & The name of the MongoDb database. \\
collection & \emph{devices} & The collection to store devices in. \\
uri & \emph{\small mongodb://localhost/?w=2} & The MongoDb connection URI. \\
\hline
\end{tabular}
\caption{Mongo Configuration Options}
\end{figure}

See the example below.

\begin{figure}[h!]
\begin{Terminal}
[mongo]
db = myapp
collection = devices
uri = mongodb://localhost/?replicaSet=myrep&journal=true&w=2&wtimeoutMS=5000
\end{Terminal}
\caption{Mongo Configuration Example}
\end{figure}

\subsection{aps}

The Apple Push Notification System requires that you use client SSL certificates to communicate with their gateway.
You will need both the private key and certificate available in PEM format to communicate with the Apple Push Notification gateway.

\textbf{TODO} - Add section on how to generate sub keys and publish to Apple.

\begin{figure}[h!]
\centering
\begin{tabular}{l l l}
\hline
Option & Value & Description \\
\hline
ssl-cert-file & \file{cert.pem} & Path to SSL certificate. \\
ssl-key-file & \file{key.pem} & Path to SSL private key. \\
sandbox & \verb|true| or \verb|false| & Connect to sandbox gateway. \\
\hline
\end{tabular}
\caption{Apple Push Notification Configuration Options}
\end{figure}

See the example below.

\begin{figure}[h!]
\begin{Terminal}
[aps]
ssl-cert-file = /etc/ssl/certs/aps.pem
ssl-key-file = /etc/ssl/private/aps.pem
sandbox = false
\end{Terminal}
\caption{Apple Push Notification Configuration Example}
\end{figure}

\subsection{c2dm}

Google's C2DM service requires an old Google Client style auth token.
You can find out more about generating one of these tokens at \url{https://developers.google.com/gdata/articles/using_cURL#authenticating-clientlogin}.

\begin{figure}[h!]
\centering
\begin{tabular}{l l l}
\hline
Option & Value & Description \\
\hline
auth-token & \verb|my_auth_token| & A Google Client Auth token. \\
\hline
\end{tabular}
\caption{Google C2DM Configuration Options}
\end{figure}

See the example below.

\begin{figure}[h!]
\begin{Terminal}
[c2dm]
auth-token = 123456789012345678901234567890
\end{Terminal}
\caption{Google C2DM Configuration Example}
\end{figure}

\subsection{gcm}

Google's GCM requires an authorization token.
You can find out more about generating one of these tokens at \url{http://developer.android.com/guide/google/gcm/gs.html}.

\begin{figure}[h!]
\centering
\begin{tabular}{l l l}
\hline
Option & Value & Description \\
\hline
auth-token & \verb|my_auth_key| & A Google Authorization Key. \\
\hline
\end{tabular}
\caption{Google GCM Configuration Options}
\end{figure}

See the example below.

\begin{figure}[h!]
\begin{Terminal}
[gcm]
auth-token = abcdefghijklmopqrstuvwxyz
\end{Terminal}
\caption{Google GCM Configuration Example}
\end{figure}

\section{REST API}

\subsection{Creating and Managing Devices}

\subsubsection{Adding a Device}

\subsubsection{Removing a Device}

\subsection{Sending Notifications}

\section{Metrics}

Many system operators will want to know what is going on with their system.
Postal provides a mechanism for being notified about events happening within the system.
This mechanism uses a \verb|PUBSUB| channel via \verb|Redis| to deliver JSON formatted events.
This feature requires that you configure Postal with \verb|--enable-redis=yes|.

\subsection{Configuration}

Metrics delivery via \verb|Redis| provides the following configuration options.

\begin{figure}[h!]
\centering
\begin{tabular}{l l l}
\hline
Option & Value & Description \\
\hline
enabled & \verb|true| or \verb|false| & If metric delivery is enabled. \\
host & \emph{hostname} & The hostname of the Redis server. \\
port & \verb|6379| & The port of the Redis server. \\
channel & \emph{event:postal} & The Redis PUBSUB channel name. \\
\hline
\end{tabular}
\caption{Redis Configuration Options}
\end{figure}

The options should be stored in the \verb|redis| group in \file{postal.conf}.
See the example below.

\begin{figure}[h!]
\begin{Terminal}
[redis]
enabled = true
host = localhost
port = 6379
channel = events:postal
\end{Terminal}
\caption{Redis Configuration Example}
\end{figure}

\subsection{Events}

The following sections describe the various types of events that you may receive.

\subsubsection{Device Added}

This event is published when Postal has received a request to add a new device.

\begin{Terminal}
{
  "Action": "device-added",
  "DeviceType": "gcm",
  "DeviceToken": "12341234-12341234",
  "User": "000000000000111122223333"
}
\end{Terminal}

\subsubsection{Device Updated}

This event is published when Postal has updated the information for a device.
This could happen if an API request has been processed to change the attributes of the device.

\begin{Terminal}
{
  "Action": "device-updated",
  "DeviceType": "gcm",
  "DeviceToken": "12341234-12341234",
  "User": "000000000000111122223333"
}
\end{Terminal}

\subsubsection{Device Removed}

This event is published when Postal has been requested to remove a device.
This could happen from an API request that specifically requests the removal of a device.
This could also happen if push notifications have been disabled for a device.
Additionally, with APS, this could happen if the feedback service has requested the removal of a device.

\begin{Terminal}
{
  "Action": "device-removed",
  "DeviceType": "gcm",
  "DeviceToken": "12341234-12341234",
  "User": "000000000000111122223333"
}
\end{Terminal}

\subsubsection{Device Notified}

This event is published when Postal has successfully delivered a notification to the gateway.
This does not mean that the device has received the notification.
The various gateways to not provide this information so that is not available.

\begin{Terminal}
{
  "Action": "device-notified",
  "DeviceType": "gcm",
  "DeviceToken": "12341234-12341234",
  "User": "000000000000111122223333"
}
\end{Terminal}

\end{document}
